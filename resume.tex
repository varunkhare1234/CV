\documentclass[a4paper,10pt]{article}
%-----------------------------------------------------------
\usepackage[top=0.75in, bottom=0.75in, left=0.55in, right=0.85in]{geometry}
\usepackage{graphicx}
\usepackage{url}
\usepackage{palatino}
\usepackage{hyperref}
\usepackage{tabularx}
\usepackage{fontawesome}
\fontfamily{SansSerif}
\selectfont

\usepackage[T1]{fontenc}
\usepackage
%[ansinew]
[utf8]
{inputenc}

\usepackage{color}
\definecolor{mygrey}{gray}{0.75}
\textheight=9.75in
\raggedbottom

\setlength{\tabcolsep}{0in}
\newcommand{\isep}{-2 pt}
\newcommand{\lsep}{-0.5cm}
\newcommand{\psep}{-0.6cm}
\renewcommand{\labelitemii}{$\circ$}

\pagestyle{empty}
%-----------------------------------------------------------
%Custom commands
\newcommand{\resitem}[1]{\item #1 \vspace{-2pt}}
\newcommand{\resheading}[1]{{\small \colorbox{mygrey}{\begin{minipage}{0.975\textwidth}{\textbf{#1 \vphantom{p\^{E}}}}\end{minipage}}}}
\newcommand{\ressubheading}[3]{
\begin{tabular*}{6.62in}{l @{\extracolsep{\fill}} r}
	\textsc{{\textbf{#1}}} & \textsc{\textit{[#2]}} \\
\end{tabular*}\vspace{-8pt}}
%-----------------------------------------------------------

\begin{document}
\begin{minipage}{0.7\textwidth}
    \Huge{Varun Khare} \\
    \normalsize{Undergraduate Computer Science}\\
\end{minipage}
\begin{minipage}{0.35\textwidth}
\begin{tabular}{m{0.5cm} l}
    \faEnvelope & \href{mailto:varun@iitk.ac.in}{varun@iitk.ac.in}  \\
    \faGithub & \href{https://www.github.com/varunkhare1234}{github.com/varunkhare1234}\\
    \faPhone & +91-8717983153
\end{tabular}{}
\end{minipage}

\resheading{\textbf{ACADEMIC DETAILS} }\\[\lsep]
\\ \\
%\begin{table}[ht!]
%\begin{center}
\indent \begin{tabular}{ l @{\hskip 0.50in} l @{\hskip 0.5in} l @{\hskip 0.5in} l }
\hline
\textbf{Examination} & \textbf{Institute} & \textbf{Year} & \textbf{CPI/\%} \\
\hline
\\
\textit{Computer Science and Engineering} & IIT Kanpur & 2015-present & 9.24\\
\textit{Class XII} & Delhi Public School, Bhopal & 2015 & 93.8\\
\textit{Class X} & Delhi Public School, Bhopal & 2013 & 10.0\\
\\
\hline
\end{tabular}
%\end{center}
%\end{table}
\\ \\
\indent \textbf{Relevant Courses}:\\ \indent Discrete Mathematics | Abstract Algebra | Logic in Computer Science\\ \indent Data Structures and Algorithms | Computer Organization | Machine Learning Theory

\resheading{\textbf{HONORS AND AWARDS} }\\[\lsep]
\\ \\
\indent \begin{tabular}{l@{\hskip 0.24in}| @{\hskip 0.1in} l @{\hskip 0.3in} l}
    Fellowships & National Talent Search Examination (\textbf{NTSE}), 2013 & Government of India\\
     &  \textbf{KVPY} scholar, 2014 & Government of India\\
     \\
     Awards & \textbf{Academic Excellence Award}, 2015-2016 & IIT Kanpur\\
     & All-India Rank \textbf{192} amongst 1.5 million students & IIT-JEE, 2015\\
     & Scholarship (Complete fee-waiver) 2013 & DPS Bhopal\\
\end{tabular}{}

\resheading{\textbf{FIELDS OF INTEREST} }\\[\lsep]
\begin{itemize}
\item Artificial Intelligence and Machine Learning
\item Computer Vision : Scene Understanding, 3D modelling, Motion Capture
\item Augmented reality and Mixed Reality
\end{itemize}

\resheading{\textbf{MAJOR PROJECTS} }\\[\lsep]
\begin{itemize}
\item \textbf{3D Reconstruction} (Research Project) \\
 \emph{(Guide:Self
, May'16 - prsent)} \\[-0.6cm]
	\begin{itemize}\itemsep \isep
	\item Objective : Understand the principles of reconstruction, epipolar geometry.
	\item Coded \textbf{8 point} algorithm in Matlab and created a meshlab model from images. Studied Multiple View Geometry by Hartley and Zisserman and Structure from motion research papers
	\item \textbf{Single View Reconstruction:} Implemented deep learning model \textbf{3DR2N2} for single object; \textbf{Scene reconstruction} using monocular cues applied on segmented super pixels. 
	\end{itemize}

\item \textbf{AR Navigation} (Programming Club Projec) \\
 \emph{(Guide: Self, May'16 - June'16)} \\[-0.6cm]
	\begin{itemize}\itemsep \isep
	\item Created \textbf{Android} navigation app using Google Directions API and \textbf{unity3d game engine}.
	\item Relayed unity graphics on camera feed according to accelerometer and gyroscope readings. GPS and compass was used to detect roads.
	\item Awarded \textbf{best club project} | \faGithub : \href{https://github.com/varunkhare1234/augmented-reality-app}{varunkhare1234/augmented-reality-app}
	\end{itemize}
\item \textbf{On Device Machine Learning} (New York Office, \textit{IIT Kanpur}) \\
 \emph{(Jan'17 - present)} \\[-0.6cm]
	\begin{itemize}\itemsep \isep
	\item Worked on \textbf{CoCoA} framework and Federated Machine Learning 
	\item Went through randomized coordinate descent, and other approaches used earlier for distributed learning.
	\end{itemize}
\item \textbf{Android App Dev} (New York Office, \textit{IIT Kanpur}) \\
 \emph{(May'16 - present)} \\[-0.6cm]
	\begin{itemize}\itemsep \isep
	\item Implemented \textbf{SSE} notifications, Search functionality, code abstraction and \textbf{app-caching}
	\item Worked on server client interaction, material UI, \textbf{data and property binding} 
	\item lead a team of 12 interns at NYO.
	\end{itemize}
\item \textbf{Cryptography} (Course Project \textit{Discrete Mathematics}) \\
 \emph{(Guide: Prof. Nitin Saxena, July'16 - Nov'16)} \\[-0.6cm]
	\begin{itemize}\itemsep \isep
	\item Implemented \textbf{Elgamal} algorithm giving a demonstration of encryption and decryption in python
	\item Read about encryption systems and other algorithms involved \textbf{RSA, ELgamal, Rabin-Miller}
	\item \textbf{Github:} \ \href{https://github.com/varunkhare1234/elgamal}{github.com/varunkhare1234/elgamal}
	\end{itemize}
\item\textbf{Other Projects}
    \begin{itemize}
    \item Mentored \textbf{Depression Therapy Chat bot} as Programming Club project. Students implemented Sentiment Analysis using twitter data-set for user response classification. Classified in \textbf{Most innovative student activities} by IITK Newsletter.
    \item Android application development for Antaragni 2016 | Mechanical Coin sorter as Technical Arts project.
    \end{itemize}
\end{itemize}	

\resheading{\textbf{TECHNICAL SKILLS} }\\[\lsep]
\\ \\
\indent \begin{tabular}{c @{\hskip 0.3in}| @{\hskip 0.1in}l}
    Languages & \textbf{Proficient}: C, Java, Matlab/Octave, Bash, python, MySQL, \LaTeX \\
     & \textbf{Experienced}:R, Verilog, Assembly, C\#, HTML\\
     \\
     Softwares & \textbf{OS}: ARCH linux, Ubuntu, Windows\\
     & \textbf{Softwares}: Android Studio, Unity 3D, NetBeans, Git, vim, CI Jenkins
\end{tabular}

\resheading{\textbf{POSITION OF RESPONSIBILITY} }\\[\lsep]
\\ \\ \indent
\begin{tabular}{l @{\hskip 0.5in}l @{\hskip 0.3in}r }
\textbf{Coordinator} & \textit{Programming Club, IIT Kanpur} & \emph{(May'17-Present)}\\
\textbf{Coordinator} & \textit{Google Developers Group} &
\emph{(May'16-April'17)}\\
\textbf{Manager} & \textit{Software Corner, Techkriti 2017 (Annual Tech Fest)} &
\emph{(May'16-April'17)}\\
\textbf{Student Guide} & \textit{Counselling service, IIT Kanpur} & \emph{(June'16-April'17)}\\
\textbf{Academic Mentor} & \textit{Counselling service, IIT Kanpur} & \emph{(June'16-April'17)}\\
\textbf{Senior Web Executive} & \textit{Antaragni 2016 (Annual Cult Fest)} & \emph{(May'16-Nov'16)}\\
\textbf{Senior Executive} & \textit{Entrepreneurship Cell, IIT Kanpur} & \emph{(June'16-April'17)}\\
\end{tabular}

\end{document}

