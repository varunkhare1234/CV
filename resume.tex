\documentclass[a4paper,10pt]{article}
%-----------------------------------------------------------
\usepackage[top=0.70in, bottom=0.70in, left=0.55in, right=0.85in]{geometry}
\usepackage{graphicx}
\usepackage{url}
\usepackage{palatino}
\usepackage{hyperref}
\usepackage{tabularx}
\usepackage{fontawesome}
\fontfamily{SansSerif}
\selectfont

\usepackage[T1]{fontenc}
\usepackage
%[ansinew]
[utf8]
{inputenc}

\usepackage{color}
\definecolor{mygrey}{gray}{0.75}
\textheight=9.75in
\raggedbottom

\setlength{\tabcolsep}{0in}
\newcommand{\isep}{-2 pt}
\newcommand{\lsep}{-0.5cm}
\newcommand{\psep}{-0.6cm}
\renewcommand{\labelitemii}{$\circ$}

\pagestyle{empty}
%-----------------------------------------------------------
%Custom commands
\newcommand{\resitem}[1]{\item #1 \vspace{-2pt}}
\newcommand{\resheading}[1]{{\small \colorbox{mygrey}{\begin{minipage}{0.975\textwidth}{\textbf{#1 \vphantom{p\^{E}}}}\end{minipage}}}}
\newcommand{\ressubheading}[3]{
\begin{tabular*}{6.62in}{l @{\extracolsep{\fill}} r}
	\textsc{{\textbf{#1}}} & \textsc{\textit{[#2]}} \\
\end{tabular*}\vspace{-8pt}}
%-----------------------------------------------------------

\begin{document}
\begin{minipage}{0.7\textwidth}
    \Huge{Varun Khare} \\
    \normalsize{Undergraduate Computer Science}\\
\end{minipage}
\begin{minipage}{0.35\textwidth}
\begin{tabular}{m{0.5cm} l}
    \faGlobe & \href{http://home.iitk.ac.in/~varun}{home.iitk.ac.in/\sim varun}\\
    \faEnvelope & \href{mailto:varun@iitk.ac.in}{varun@iitk.ac.in}  \\
    \faGithub & \href{https://www.github.com/varunkhare1234}{github.com/varunkhare1234}\\
    \faPhone & +91-8717983153
\end{tabular}{}
\end{minipage}

\resheading{\textbf{ACADEMIC DETAILS} }\\[\lsep]
\\ \\
%\begin{table}[ht!]
%\begin{center}
\indent \begin{tabular}{ l @{\hskip 0.50in} l @{\hskip 0.5in} l @{\hskip 0.5in} l }
\hline
\textbf{Examination} & \textbf{Institute} & \textbf{Year} & \textbf{CPI/\%} \\
\hline
\\
\textit{Computer Science and Engineering} & IIT Kanpur & 2015-present & 9.0\\
\textit{Class XII} & Delhi Public School, Bhopal & 2015 & 93.8\\
\textit{Class X} & Delhi Public School, Bhopal & 2013 & 10.0\\
\\
\hline
\end{tabular}
%\end{center}
%\end{table}
\\ \\
\indent \textbf{Relevant Courses}:
\\ \\
\indent \begin{tabular}{l@{\hskip 0.24in}|@{\hskip 0.1in} l @{\hskip 0.3in}|@{\hskip 0.1in} l}
Computational Cognitive Science & Stochastic Processes & Discrete Mathematics\\
Bayesian Machine Learning & Introduction to Machine learning& Data Structures and Algorithms\\
Learning Theory  &  Probability and Statistics
& Theory of Computation\\
\end{tabular} \\ \\
\indent \resheading{\textbf{HONORS AND AWARDS} }\\[\lsep]
\\ \\
\indent \begin{tabular}{l@{\hskip 0.24in}| @{\hskip 0.1in} l @{\hskip 0.3in} l}
    Fellowships & National Talent Search Examination (\textbf{NTSE}), 2013 & Government of India\\
     &  \textbf{KVPY} scholar, 2014 & Government of India\\
     \\
     Awards & \textbf{Academic Excellence Award}, 2015-2016 & IIT Kanpur\\
     & All-India Rank \textbf{40} amongst 1.5 million students & IIT-MAINS, 2015\\
     & All-India Rank \textbf{192} amongst 150k students & IIT-JEE, 2015\\
     & Scholarship (Complete fee-waiver) 2013 & DPS Bhopal\\
\end{tabular}

\indent \resheading{\textbf{FIELDS OF INTEREST} }\\[\lsep]
\begin{itemize}
\item Neuroscience, Cognitive Sciences , Meta-Learning and psychology 
\item Augmented reality, 3D computer vision , Analysis of learning algorithms, Probabilistic modelling
\end{itemize}

\resheading{\textbf{WORK EXPERIENCE} }\\[\lsep]
\begin{itemize}
\item \textbf{Visiting Research Scholar} (National University Singapore)\\
 \emph{(Guide: Prof. Tat Seng Chua, May'18 - July'18)} \\[-0.6cm]
	\begin{itemize}\itemsep \isep
	\item \textbf{Objective} : 3D scene manipulation for Augmented Reality Systems
	\item Proposed a novel end-to-end architecture consisting of two modules for robust pose prediction and classification based on \textbf{Marr's theory of vision} 
	\item One sub module learns to \textbf{reconstruct 3D model} in its canonical viewpoint via \textbf{multi-task learning DNN}s. Another NN sub module uses \textbf{Faster R-CNN} style anchor boxes to predict the \textbf{6 DoF} poses in \textbf{continuous domain} \textbf{outperforming} PnP based solvers due to robust multi-class classification of 3D objects
	\item Also implemented texture mapping from image to corresponding aligned 3D model
	\end{itemize}

\item \textbf{Software Lead} (New York Office, IIT Kanpur)\\
 \emph{(Guide: Prof. Manindra Agarwal, May'16 - May'18)} \\[-0.6cm]
	\begin{itemize}\itemsep \isep
		\item \textbf{Objective} : Industrial grade development of ML backend and android application for NYO
	\item \textbf{ML systems}: Collaborative Filtering for \textbf{Recommendation engine}; document clustering and ranking for \textbf{query-search}
	\item \textbf{Android app}: REST APIs, SSE notifications, app-caching, Continuous integration with Jenkins, \textbf{data and property binding} and app designing 
	\item Lead a team of 16 people at NYO.
	\end{itemize}
\end{itemize}

\resheading{\textbf{MAJOR PROJECTS} }\\[\lsep]

\begin{itemize}
\item \textbf{Zero-Shot Learning Framework} (Under Graduate Project)\\
 \emph{(Guide: Prof. Piyush Rai, Jan'18 - present)} \\[-0.6cm]
	\begin{itemize}\itemsep \isep
	\item Implemented \textbf{hubness reduction} and \textbf{domain adaptation} techniques to extend the  framework published in ECML paper | \faGlobe \ \href{https://arxiv.org/pdf/1711.05820.pdf}{\textit{link}} 
	\item Reformulated the Bayesian model to incorporate neural networks for \textbf{simultaneous feature learning} and \textbf{clustering} achieving performance higher than the  \textbf{current SOTA} 
    \item Expecting to publish our results in 2019.
	\end{itemize}
\item \textbf{Adversarial Corruption in deep Neural Networks} \\
 \emph{(Guide: Prof. Purushottam Kar, Jan'18 - April'18)} \\[-0.6cm]
	\begin{itemize}\itemsep \isep
	\item \textbf{Objective} : Robust learning of deep networks
	\item Came up with a \textbf{theoretically sound} algorithm for robustly training a single hidden layer ReLU network. Practically compared the training procedure to SGD as a proof of concept.
	\item Literature survey included robust statistics, convergence analysis of two layer network and various convergence proof techniques amongst others.
	\item \textbf{Project Report:}\ \href{http://home.iitk.ac.in/~varun/CS777.pdf}{\textit{\faGlobe \ here}}
	\end{itemize}
\item \textbf{Language Understanding and Information Retrieval} (Under Graduate Project) \\
 \emph{(Guide: Prof. Arnab Bhattacharya \& Prof. Amay Karkare, July'17 - Dec'17)} \\[-0.6cm]
	\begin{itemize}\itemsep \isep
	\item \textbf{Objective} : Creating a word problem solver for elementary maths problems
	\item Implemented the model using word2vec, \textbf{co-reference resolution}, \textbf{syntactic parsing} and \textbf{dependency parsing}
	\item Involved natural language understanding,\textbf{ world concept graph generation}, quantity association and query evaluation
	\item \textbf{Github} \faGithub  : \href{https://github.com/varunkhare1234/word_problem_solver }{github.com/varunkhare1234/word\_problem\_solver}
	\end{itemize}

\item \textbf{Augmented Reality Navigation} (Programming Club Project) \\
 \emph{(Guide: Self, May'16 - June'16)} \\[-0.6cm]
	\begin{itemize}\itemsep \isep
	\item Created \textbf{Android} navigation app using Google Directions API and \textbf{unity3d game engine}.
	\item Relayed unity graphics on camera feed according to accelerometer and gyroscope readings. GPS and magnetic compass was used to detect roads.
	\item Awarded \textbf{best club project} | \faGithub : \href{https://github.com/varunkhare1234/augmented-reality-app}{varunkhare1234/augmented-reality-app}
	\end{itemize}

\item\textbf{Other Projects}
    \begin{itemize}
    \item Mentored \textbf{Depression Therapy Chat bot} as Programming Club project. Students implemented Sentiment Analysis using twitter data-set for user response classification. The model responded according a dialogue tree based on the sentiment predicted. Classified in \textbf{Most innovative student activities} by IITK Newsletter.
    \item Android application development for Antaragni 2016 | Mechanical Coin sorter as Technical Arts project.
    \end{itemize}
\end{itemize}	

\resheading{\textbf{TECHNICAL SKILLS} }\\[\lsep]
\\ \\
\indent \begin{tabular}{c @{\hskip 0.3in}| @{\hskip 0.1in}l}
    Languages & \textbf{Proficient}: Kotlin,C,C++, Java, Matlab/Octave, Bash, python, MySQL, \LaTeX \\
     & \textbf{Experienced}:R, Verilog, Assembly, C\#, HTML\\
     \\
     Softwares & \textbf{OS}: ARCH linux, Ubuntu, Windows\\
     & \textbf{Libraries and Softwares}: Tensorflow, Pytorch, Edward, Android Studio, blender, CI Jenkins
\end{tabular}\\ \\

\resheading{\textbf{POSITION OF RESPONSIBILITY} }\\[\lsep]
\\ \\ \indent
\begin{tabular}{l @{\hskip 0.5in}l @{\hskip 0.3in}r }
\textbf{Course Project Mentor} & \textit{Introduction To Machine Learning(CS771), IITK} & \emph{(June'18-Nov'18)}\\
\textbf{Coordinator} & \textit{Programming Club, IIT Kanpur} & \emph{(May'17-March'18)}\\
\textbf{Coordinator} & \textit{Google Developers Group} &
\emph{(May'16-April'17)}\\
\textbf{Manager} & \textit{Software Corner, Techkriti 2017 (Annual Tech Fest)} &
\emph{(May'16-April'17)}\\
\textbf{Student Guide} & \textit{Counselling service, IIT Kanpur} & \emph{(June'16-April'17)}\\
\textbf{Academic Mentor} & \textit{Counselling service, IIT Kanpur} & \emph{(June'16-April'17)}\\
\textbf{Senior Web Executive} & \textit{Antaragni 2016 (Annual Cult Fest)} & \emph{(May'16-Nov'16)}\\
\textbf{Senior Executive} & \textit{Entrepreneurship Cell, IIT Kanpur} & \emph{(June'16-April'17)}\\
\textbf{Secretary} & \textit{Programming Club, IIT Kanpur} & \emph{(June'16-April'17)}\\
\end{tabular}

\end{document}
